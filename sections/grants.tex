\section*{Grant Proposals}

\subsection*{Funded}

\begin{enumerate}
  \item MURI Center for Material Failure Prediction Through Peridynamics. Air Force Office of Scientific Research, 2013-2018. ONRBAA12-020, \textit{Co-PI} Total Award {\$}7,500,000.  Foster Award: \$959,153.
  \item Statistical coarse-graining of molecular dynamics into peridynamics. \textit{Subaward} from Army Reasearch Laboratories Materials in Extreme Dynamic Environments Cooperative Research Agreement.  The Johns Hopkins University, 2014.  \$101,306.
  \item Predictive simulation of material failure using peridynamics-advanced constitutive modeling, verification, and validation. Air Force FY2013 Young Investigator Program. BAA-AFOSR-2012-0001, AFOSR, 2013-2015. \textit{PI} \$360,000.
  \item Statistical coarse-graining of molecular dynamics into peridynamics. \textit{Subaward} from Army Reasearch Laboratories Materials in Extreme Dynamic Environments Cooperative Research Agreement.  The Johns Hopkins University, 2013.  \$97,471.
  \item Peridynamic simulation of pressure-shear experiments on granular media.  Sandia National Laboratories, 2013. \textit{PI} \$29,071
  \item Statistical coarse-graining of molecular dynamics into peridynamics. \textit{Subaward} from Army Reasearch Laboratories Materials in Extreme Dynamic Environments Cooperative Research Agreement.  The Johns Hopkins University, 2012.  \$91,125.
  \item Fracture Design, Placement And Sequencing In Horizontal Wells. Joint proposal with The University of Texas at Austin in response to solicitation number DE-FOA-0000724.  DOE, 2012-2015. \textit{co-PI} Total Award: {\$1,592,451}, Foster Award: \$275,250.
  \item Application of Peridynamics to Hydraulic Fracture Modeling. The University of Texas at San Antonio -- Office of the Vice President for Research, 2012. \textit{PI} \$18,927.
  \item Peridynamic Simulation of Granular Materials Undergoing Shock Compression.  Sandia National Laboratories, 2012. \textit{PI} \$32,597
  \item Sandia X-Prize Necking Challenge.  Sandia National Laboratories, 2012. \textit{PI} \$44,700.
\end{enumerate}

\subsection*{Pending}

\begin{enumerate}
  \item DTRA Young Investigator Program: Multiscale peridynamic simulation of geomaterials under impact loading. Defense Threat Reduction Agency, 2014-2016. \textit{PI} Reqesting: \$200,000.
  \item Bridging the length scales through a unified nonlocal multiscale framework. National Science Foundation, 2014-2017. \textit{PI} requesting \$234,407.
\end{enumerate}

\subsection*{Unfunded}

\begin{enumerate}
  \item DOE Career: Nonlocal porous flow in evolving fractured media using peridynamic theory. Department of Energy, 2014-2017. \textit{PI} Requesting: \$750,000. 
    \item BRIGE: A nonlocal mixture theory approach to fluid driven fracture with applications in energy production and environmental assessment. National Science Foundation, 2013-2015. Requested \$174,702.
    \item  Investigating Cellular And Subcellular Behaviors and Metabolic Mechanisms Using Thermal and Raman Imaging Techniques, National Science Foundation, 2013-2016.  Co-PI Requesting: \$705,052
    \item Dynamic Failure Mechanisms of Advanced Fiber Materials. Joint proposal with SwRI to the SwRI/UTSA CONNECT program, 2013. Requested \$99,940.
  \item Towards exascale computational mechanics: exploiting the newest generation of heterogenous HPC clusters. Oak Ridge Associated Universities Ralph E. Powe Junior Faculty Enhancement Award. Oak Ridge National Laboratories. Requesting: \$10,000. 
    \item DOE Career: Nonlocal porous flow in evolving fractured media using peridynamic theory. Department of Energy, 2013-2016. Requesting: \$749,875. 
    \item BRIGE: Identification and Simulation of Non-Local Effects to Improve Predictive Analysis of Heterogenous Materials. National Science Foundation, 2012-2014. Requested \$174,805.
    \item Discontinuous Flow and Angled Localization: Modern Challenges in Material Failure. Joint proposal with SwRI to the SwRI/UTSA CONNECT program, 2012. Requested \$93,780.
    \item A novel torsional Kolsky bar for testing materials at constant shear strain rates. Haythornthwaite Research Initiation Grant Program, 2011. Requested \$13,388.
    \item Joint proposal with SwRI in response to BAA AFOSR 2011-06 on University Center of Excellence: High-rate Deformation Physics of Heterogeneous Materials. AFOSR, 2011. Total Proposed: \$5,000,000, Foster Requested: \$377,518.
\end{enumerate}
