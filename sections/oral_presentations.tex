\section*{Presentations}

%\subsection*{Conferences}
%
%\begin{itemize}
%    \item ``Regularizing numerical simulations of shear-banding using a peridynamics-based plasticity formulation.'' (with Md.I.H.~Kahn). ASME 2014 International Mechanical Engineering Congress and Exposition. November 2014.
%    \item ``An Ordinary State Based Plasticity Model For Peridynamics.'' (with J.A.~Mitchell). ASME 2014 International Mechanical Engineering Congress and Exposition. November 2014.
%    \item ``Fracture in plates and shells with peridynamic non-ordinary state-based models.''  Meshfree Methods for Large-Scale Computational Science and Engineering. October 2014.
%    \item ``An Overview of the Progress of Meshfree Particle Methods: From SPH to EFG to RKPM to Meshfree Peridynamics.'' (with W.K~Liu, M.~Bessa). Meshfree Methods for Large-Scale Computational Science and Engineering. October 2014.
%    \item ``A nonlocal poroelastic approach to fluid driven fracture.'' (with J.~York, A.~Katiyar, H.~Ouchi, M.~Sharma). World Congress on Computational Mechanics XI.  July 2014.
%    \item ``Reproducing Continuum Dynamics''. (with M.~Bessa, W.K.~Liu, T.~Belytschko). World Congress on Computational Mechanics 2014.  July 2014.
%    \item ``A nonlocal poroelastic approach to fluid driven fracture.'' (with J.~York, A.~Katiyar, H.~Ouchi, M.~Sharma). US National Congress on Theoretical and Applied Mechanics.  June 2014.
%    \item ``Bridging the length scales by linking the atomistic model with coarser peridynamic models through molecular dynamics simulation of Polyethylene''. (with R.~Rahman). Mach Conference 2014.  April 2014.
%    \item ``Regularizing numerical simulations of strain-localization using a peridynamics-based plasticity formulation''. (with Md.I.~Kahn, D.J.~Littlewood, and J.A.~Mitchell). International Workshop on Computational Mechanics of Materials, IWCMM XXIII. October 2013
%    \item ``A non-local formulation for fluid flow and mass transport in porous media based on peridynamic theory''. (with A.~Katiyar and M.~Sharma). 12th US National Congress on Computational Mechanics. July 2013.
%    \item ``A novel hierarchical multiscale modeling framework for polyethylene systems using Peridynamics and molecular dynamics''. (with R. Rahman). 2013 Mach Conference, Annapolis, MD. April 2013. 
%    \item ``Two-Dimensional Semi-Analytic Solutions to the Linearized State-Based Peridynamic Equilibrium Equation''. (with J.T. O'Grady). USACM Workshop on Nonlocal Damage and Failure: Peridynamics and other nonlocal methods. March 2013.
%    \item ``A Peridynamics Based Hierarchical Multiscale Modeling Framework Between Continuum and Atomistic Scales''. (with R. Rahman, A. Haque). USACM Workshop on Nonlocal Damage and Failure: Peridynamics and other nonlocal methods. March 2013.
%    \item ``Lessons Learned in Modeling Ductile Failure with Peridynamics''. (with D.J. Littlewood). USACM Workshop on Nonlocal Damage and Failure: Peridynamics and other nonlocal methods. March 2013.
%    \item ``A Peridynamics Formulation of the Coupled Mechanics-Fluid Flow Problem''. (with A. Katiyar, H. Ouchi, M.M. Sharma). USACM Workshop on Nonlocal Damage and Failure: Peridynamics and other nonlocal methods. March 2013.
%    \item ``Implicit time integration of an ordinary state-based peridynamic plasticity model with isotropic hardening.'' (with D.J. Littlewood, J.A. Mitchell, M.L. Parks).  ASME IMECE 2012.  November 2012.
%    \item ``Implicit time integration of an ordinary state-based peridynamic plasticity model with isotropic hardening.'' (with D.J. Littlewood, J.A. Mitchell, M.L. Parks).  SiViRT Simulation and Vizualization Symposium.  November 2012.
%    \item ``Peridynamic Modeling of Localization in Ductile Metals.'' (with D.J. Littewood and B.L. Boyce)  International Workshop on Computational Mechanics of Materials,
%IWCMM XXII. September 2012
%    \item ``Viscoplasticity using peridynamics.''  (with S.A. Silling and W. Chen) 10th US National Congress on Computational Mechanics. July 2009.
%\end{itemize}

%\subsubsection*{Student Delivered}
%
%\begin{itemize}
%  \item ``Peridynamic beams, plates, and shells: a non-ordinary state-based model.'' (with J.~O'Grady). ASME 2014 International Mechanical Engineering Congress and Exposition. November 2014.
%  \item ``Peridynamic beams, plates, and shells: a non-ordinary state-based model.'' (with J.~O'Grady). Society of Engineering Science 2014. October 2014.
%  \item ``The Next Generation Model for Predicting the Growth of Complex Fracture Networks.'' (with J.R.~York). 2014 Hydraulic Fracturing and Sand Control Joint Industry Program Technical Review.  April 2014.
%  \item ``A peridynamic model of diffusive fluid flow through a deformable media.'' (with J.R. York). 2013 SACNAS National Conference. October 2013.
%  \item ``A complex-step method for tangent-stiffness calculation in a massively parallel computational peridynamics code.'' (with M.D.~Brothers and H.R.~Millwater). 12th US National Congress on Computational Mechanics. July 2013.
%  \item ``Intragranular fracture and frictional effects in granular materials under pressure-shear loading.'' (with A.M. Peterson and T.J. Vogler) 18th Biennial Intl. Conference of the APS Topical Group on Shock Compression of Condensed Matter held in conjunction with the 24th Biennial Intl. Conference of the Intl. Association for the Advancement of High Pressure Science and Technology (AIRAPT). July 2013.
%\end{itemize}

\subsection*{Technical Talks}

\begin{itemize}
	\item ``Geometry, AI and the Brain.'' SMaSH Symposium, Harvard University, May 2022.
	\item ``Geometry, AI and the Brain.'' NeuroTheory Symposium, Harvard University, March 2022.
	\item ``Sparse Coding, Artificial Neural Networks, and the Brain: Toward Substantive Intelligence.'' Center for Brain Science, Harvard University, April 2021.
	\item ``Sparse Coding, Artificial Neural Networks, and the Brain: Toward Substantive Intelligence.'' Institute for Artificial Intelligence and Fundamental Interactions Colloquium Series, MIT, April 2021.
	\item ``Interpretable AI in Computational Neuroscience: Sparse Coding, Artificial Neural Networks, and the Brain.'' Princeton Neuroscience Institute, Princeton University, April 2021.
	\item ``Interpretable AI in Computational Neuroscience: Sparse Coding, Artificial Neural Networks, and the Brain.'' Center for Theoretical Neuroscience, Columbia University, March 2021.
	\item ``Interpretable AI in Computational Neuroscience: Sparse Coding, Artificial Neural Networks, and the Brain.'' Computer Science Colloquium, Indiana University Bloomington, March 2021.
	\item ``Deeply-sparse signal representations.'' Physical Mathematics Seminar Series, MIT, February 2021.
	\item ``Interpretable AI in Computational Neuroscience: Sparse Coding, Artificial Neural Networks, and the Brain.'' Center for Computational Neuroscience, University of Washington, January 2021.
	\item ``Deeply-sparse signal representations.'' Conference of the Mathematical Theory of Deep Neural Networks (DeepMath), November 2020.
	\item ``Learning deeply-sparse signal representations.''  6.S975 Seminar Series (Advanced Topics in Signal Processing), MIT, February 2020.
	\item ``Auto-encoders for convolutional dictionary learning.'' IBRO-SIMONS Computational Neuroscience Imbizo, Muizenberg, South Africa, January 2020.	
	\item ``AI in computational neuroscience: sparsity, artificial neural networks and the brain.'' Center for Mind, Brain, Computation and Technology Seminar Series, Stanford University, October 2019.
	\item ``Learning deeply-sparse signal representations.''  Applied Mathematics Seminar Series, Tufts University, September 2019.	
	\item ``Population codes, hierarchical sparse coding and connections to artificial neural networks.''  Oxford University Cortexlab Seminar Series, Oxford UK, May 2019.
	\item ``AI in computational neuroscience: sparsity, artificial neural networks and the brain.''  International Brain Lab (IBL) annual meeting, Paris France, May 2019.
	\item ``Learning deeply-sparse signal representations.''  Electrical Engineering Seminar Series, Cornell Tech, April 2019.
	\item ``AI in computational neuroscience: sparsity, artificial neural networks and the brain.''  Amazon AWS AI in Practice -- DSP, Audio, Speech and Languages, Electrical Engineering Seminar Series, Palo Alto CA, March 2019.
	\item ``Learning deeply-sparse signal representations.''  Electrical Engineering Seminar Series, Rice University, March 2019.
	\item ``Learning deeply-sparse signal representations.''  Electrical Engineering Seminar Series, Harvard University, March 2019.
	\item ``Sparse coding, sensory processing in the brain, and artificial neural networks.'' IBRO-SIMONS Computational Neuroscience Imbizo, Muizenberg, South Africa, January 2019.
	\item ``Estimating a separable random field from binary observations.'' Department of ECE, University of Maryland -- College Park, March 2017.
    \item ``Estimating a separable random field from binary data.'' Center for Brain Science, Harvard University, November 2016.
    \item ``Estimating a separable random field from binary data.'' Department of ECE, SILO Seminar Series, University of Wisconsin -- Madison, October 2016.
   	\item ``Estimating structured state-space models from point-process data.'' Neurocontrol Workshop, Automatic Control Conference, Boston MA, July 2016.
    \item ``Estimating structured state-space models from point-process data.'' Second Workshop on Modelling Neural Activity, Waikoloa HI, June 2016.
    \item ``New time-frequency tools toward a more precise characterization of rhythms from the brain.'' Institute of Applied and Computational Sciences, Harvard University, February 2016.
    %\item ``Estimating structured time-frequency representations by iteratively re-weighted least squares.'' Department of EECS, University of Michigan -- Ann Arbor, .
%    \item ``Nonlocal multiphysics for heterogeneous materials, anomalous diffusion, and fracture.'' The University of Texas at Austin, Department of Engineering Mechanics, September 2014.
%    \item ``Nonlocal multiphysics for heterogeneous materials, anomalous diffusion, and fracture.'' ExxonMobil - Corporate Strategic Research, July 2014.
%    \item ``A model for nonlocal diffusion and fluid-driven fracture.'' USACM/IUTAM Symposium on Connecting Multiscale Mechanics to Complex Material Design. Northwestern University. May 2014.
%    \item ``Nonlocal multiphysics for heterogeneous materials, anomalous diffusion, and fracture.'' The University of Texas at Austin, Department of Petroleum \& Geosystems Engineering. March 2014.
%    \item ``Nonlocal multiphysics for heterogeneous materials, anomalous diffusion, and fracture.'' Northwestern University, Department of Mechanical Engineering. January 2014.
%    \item ``Peridynamics as a unified theory for heterogenous media, anomalous porous flow, and fracture.'' The University of Texas at Austin, Department of Petroleum \& Geosystems Engineering. October 2013.
%    \item ``Unifying the mechanics of continuous and discontinuous media.''  Army Research Laboratory.  February 2013.
%    \item ``Unifying the mechanics of continuous and discontinuous media.''  The Johns Hopkins University, Center for Advanced Ceramics and Metallic Systems.  July 2012.
%    \item ``Unifying the mechanics of continuous and discontinuous media.''  Texas Tech University, Mechanical Engineering.  April 2012.
%    \item ``Hydraulic fracturing and its environmental impact: a short address of major public concerns.'' Presentation for the Center for Simulation, Visualization, and Real-Time Prediction participation in UTSA Earthweek 2012.  April 2012.
%    \item ``Unifying the mechanics of continuous and discontinuous media.''  2011 International Workshop on Intensive Loading and its Effects.  State Key Laboratory of Explosion Science and Technology, Beijing Institute of Technology.  Beijing, China. December 2011.
%
%    \item ``Peridynamic modeling of viscoplasticity and dynamic fracture.''  University of Nebraska, Engineering Mechanics. April 2010.
%
%    \item ``Peridynamic modeling of viscoplasticity and dynamic fracture.''  University of New Mexico, Mechanical Engineering. February 2010.
\end{itemize}

%\subsection*{Non-technical Talks}
%
%\begin{itemize}
%	\item ``Why and How to Leverage Amazon Cloud Services to Deploy JupyterHub at Scale?'' Keynote, Jupytercon, August 2017 (with Faras Sadek).
%    \item ``Labz `n da wild: teaching signal processing using wearables and jupyter notebooks in the cloud.'' Scientific Computing with Python 2016 (Scipy 2016), July 2016 (with Faras Sadek, Yasha Iravantchi and Yingzhuo (Diana) Zhang).
%    \item ``Wearable signal processing using docker notebook containers on AWS.'' Jupyter Day Boston, Harvard University, February 2016 (with Faras Sadek, Yasha Iravantchi and Yingzhuo (Diana) Zhang).
%\end{itemize}


%\subsection*{Poster}
%
%\begin{itemize}
%  \item ``Intragranular fracture and frictional effects in granular materials under pressure-shear loading.'' (with A.M. Peterson and T.J. Vogler) 18th Biennial Intl. Conference of the APS Topical Group on Shock Compression of Condensed Matter held in conjunction with the 24th Biennial Intl. Conference of the Intl. Association for the Advancement of High Pressure Science and Technology (AIRAPT). July 2013.
%\end{itemize}
%
%
